%!TEX TS-program = pdflatex % pdflatex으로 컴파일하도록 지정
%!TEX encoding = UTF-8 Unicode % UTF-8 인코딩 사용

\documentclass{article} % article 클래스 사용

% 필요한 패키지 로드
\usepackage[utf8]{inputenc} % UTF-8 인코딩 처리 (pdflatex용)
\usepackage[T1]{fontenc} % 글꼴 인코딩 설정
\usepackage[english]{babel} % 언어 설정 (필요에 따라 ko 등으로 변경 가능하나 CV는 보통 English)
\usepackage{geometry} % 여백 설정 패키지
\usepackage{hyperref} % 하이퍼링크 (이메일, URL) 설정 패키지
\usepackage{enumitem} % 리스트(itemize, enumerate) 간격 조절 패키지
\usepackage{ragged2e} % 텍스트 정렬 패키지 (\RaggedRight 사용)
\usepackage{xcolor} % 색상 사용 패키지 (줄 색상 등을 조절하고 싶을 때)
\usepackage{tikz} % 줄 스타일 등 고급 조절 시 사용 가능 (여기서는 간단히 hrule 사용)

% 여백 설정: 좌, 우, 상, 하 여백을 동일하게 2.5cm로 설정 (원하는 값으로 조절)
\geometry{a4paper, margin=2.5cm}

% 하이퍼링크 색상 설정 (선택 사항)
\hypersetup{
    colorlinks=true,
    urlcolor=blue,
    linkcolor=blue,
    citecolor=blue
}

% 전체 텍스트를 왼쪽 정렬 (CV에서 흔히 사용)
\RaggedRight

%-------------------------------------------------------------------------------
% 기본적인 CV 항목(cventry와 유사)을 위한 사용자 정의 명령어 정의
% \mycventry{직책/역할}{소속}{위치}{기간}{세부 내용 (itemize 환경에 들어갈 \item{} 들)}
%-------------------------------------------------------------------------------
\newcommand{\mycventry}[5]{
  \vspace{1.5em} % <--- 각 항목(cventry) 사이의 간격. 이 값을 조절.

  \textbf{#1}, #2 \\ % 직책/역할 (굵게), 소속
  #3 $|$ #4 \\[0.4em] % <--- ##### 이 부분이 Location/Dates와 Bullet Point 사이 간격 #####
                     % <--- ##### [0.4em] 값을 조절하여 원하는 간격으로 설정하세요. #####
                     % 값을 늘리면 간격이 넓어지고 줄이면 좁아집니다.

  % 세부 내용 리스트 (itemize 환경)
  % nosep: 항목 사이의 기본 간격을 없앰
  % after: itemize 환경 자체의 아래 여백을 조절
  \begin{itemize}[nosep, after=\vspace{-0.4em}] % before=\vspace{-0.4em}는 삭제하여 Location/Dates와의 기본 간격 확보
    #5 % \item{} 으로 구성된 세부 내용들
  \end{itemize}
}

%-------------------------------------------------------------------------------
% 섹션 나누는 줄 정의
%-------------------------------------------------------------------------------
\newcommand{\sectionsplitrule}{
%  \vspace{1.5em} % <--- ##### 줄 위에 넣을 수직 간격 #####
  \hrule % <--- 수평선 그리기 명령. 기본 두께와 색상 사용.
         % 만약 색상이나 두께 조절 필요 시 \textcolor{색상이름}{\rule{\linewidth}{선두께}} 등으로 대체 가능
  \vspace{1em} % <--- ##### 줄 아래에 넣을 수직 간격 #####
}


%-------------------------------------------------------------------------------
% 문서 시작
%-------------------------------------------------------------------------------
\begin{document}

%===============================================================================
% 개인 정보 (상단 헤더 역할)
% 필요에 따라 추가/수정/삭제하세요.
%===============================================================================
\begin{center}
    {\Huge \textbf{Jaewon Jang}} % 이름 (크고 굵게)

    \vspace{0.5em} % 이름과 정보 사이 간격

    Master student, \href{https://sites.google.com/view/ddfegroup/home}{DDFE LAB} \\% 홈페이지 | 연구실 페이지 (하이퍼링크)
\ % 직책/소속
    100, Inha-ro, Michuhol-gu, Incheon, Republic of Korea \\ % 주소
    (+82) 10-3193-5624 $|$ \href{mailto:poroparo5624@gmail.com}{poroparo5624@gmail.com} \\ % 전화 | 이메일 (하이퍼링크)
    \href{https://github.com/VortexyAether}{github} $|$ \href{https://www.linkedin.com/in/jaewon-jang-895785252}{linkedin} \\ % GitHub | LinkedIn (하이퍼링크)
    \href{https://vortexyaether.github.io}{Webpage} 
\end{center}


%------------------------------------------------------

%------------------------------------------------------

\section*{Education}
\sectionsplitrule
%\vspace{0.5em} % 섹션 제목과 내용 사이 간격

% my cventry 명령어 사용
\mycventry
  {Undergraduated student} % 직책/역할
  {INHA University} % 소속
  {Incheon, S.Korea} % 위치
  {Mar. 2018 - Feb. 2025} % 기간
  { % 세부 내용 (itemize의 \item{} 들)
    \item Mechanical Engineering 
    \item GPA: 3.95 / 4.5
    \item Major GPA: 4.12 / 4.5
    \item Total Credits: 150 
  }

\mycventry
  {Master student}
  {INHA University} % 소속
  {Incheon, S.Korea} % 위치
  {Mar. 2025 - Present} % 기간
  {
    \item Mechanical Engineering 
  }


%-------------------------------------------------------------------------------
% Experience 섹션 예시
% 앞에서 알려주신 내용을 바탕으로 작성
%-------------------------------------------------------------------------------
\section*{Experience}
\sectionsplitrule

% 학부 연구 경험
\mycventry
  {Undergraduated student} % 직책/역할 (awesome-cv와 동일하게 사용)
  {INHA University} % 소속
  {Incheon, S.Korea} % 위치
  {Mar. 2018 - Feb. 2025} % 기간
  { % 세부 내용 (\item{} 으로 작성)
    \item Object classification (human, vehicle, and labeled/unlabeled wastes) with YOLO
	\item PEMFC 0-dimensional modeling with MATLAB - calculates Ohmic, Concentration losses
	\item Convolutional Neural Network based Dynamometer Driver - Simulating and performance checking with WLTC cycle
	\item Pollution separate in pipe fluids - Construct IOT sensor system with Node-RED
    \item Physics-informed Neural Networks for prediction of steady-states Airfoil flowfields
  }

% 석사 연구 경험
\mycventry
  {Master student} % 직책/역할
  {DDFE Lab. (INHA University)} % 소속
  {Incheon, S.Korea} % 위치
  {Mar. 2025 - Present} % 기간
  { % 세부 내용
    \item Generative AI for prediction of wafer-scale process non uniformity @SAMSUNG Inc.
  }


\mycventry
  {Squad leader} % Job title
  {Ground Operations Command at Republic of Korea Army} % Organization
  {Yongin, S.Korea} % Location
  {Feb. 2020 - Aug. 2021} % Date(s)
  {
    \item 20mm Vulcan Anti-Aircraft Gun Operator \& Administrative Clerk
  }

%-------------------------------------------------------------------------------
% Skills section 
%-------------------------------------------------------------------------------
\section*{Skills}
\sectionsplitrule

\begin{itemize}[leftmargin=1.5em, parsep=0.5em] % parsep으로 항목 사이 간격 조절 가능
    \item \textbf{Programming Languages:} Python, MATLAB, C++ (Basic)
    \item \textbf{Frameworks/Libraries:} PyTorch, scikit-learn, NumPy, Node-RED
    \item \textbf{Tools:} LaTeX, OpenFOAM, PointWise, 
\end{itemize}


\section*{Awards and Honors}
\sectionsplitrule

\begin{itemize}[leftmargin=1.5em, parsep=0.5em] % parsep으로 항목 사이 간격 조절 가능
    \item 2022-1 Vertically Integrated Projects, Grand prize, Inha University (Jun. 2022)
    \item 2022-2 Vertically Integrated Projects, Grand prize, Inha University (Dec. 2022)
    \item 2022 Inno Think Mackathon, Grand prize, POSCO EnC (Jan. 2023)
    \item 2024 ICT Convergence Project Competition, Participation Award, NTREX (Jul. 2024)
\end{itemize}

%-------------------------------------------------------------------------------
% 추가 섹션 예시 (필요에 따라 복사하여 사용)
% \section*{Projects}
% ... 내용 ...
% \sectionsplitrule
%
% \section*{Awards \& Honors}
% ... 내용 ...
% \sectionsplitrule
%-------------------------------------------------------------------------------


%===============================================================================
% 문서 끝
%===============================================================================
\end{document}
